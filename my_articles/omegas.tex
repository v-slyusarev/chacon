\documentclass[a4paper]{article}
\usepackage[margin=1.0in]{geometry}

\usepackage[l2tabu,orthodox]{nag}
\usepackage{indentfirst}
\usepackage{amssymb,amsfonts}
\usepackage[]{mathtools}
\usepackage{cmap}
\usepackage[T2A]{fontenc}
\usepackage[utf8]{inputenc}
\usepackage{ucs}
\usepackage[russian,english]{babel}
\usepackage[babel = true]{microtype}
\usepackage{graphicx}
\usepackage[colorinlistoftodos, textsize=tiny]{todonotes}

\usepackage{color}
\definecolor{darkblue}{rgb}{0,0,.75}
\definecolor{darkred}{rgb}{.7,0,0}
\definecolor{darkgreen}{rgb}{0,.7,0}

\usepackage[
    draft = false,
    unicode = true,
    colorlinks = true,
    allcolors = blue,
    hyperfootnotes = true
]{hyperref}
\usepackage{amsmath}
\usepackage{amsthm}
    \theoremstyle{plain}
    \newtheorem{theorem}{Theorem}
    \newtheorem{task}{Задача}
    \newtheorem{lemma}{Lemma}
    \newtheorem{proposition}{Proposition}
    \newtheorem{corollary}{Corollary}
    \theoremstyle{definition}
    \newtheorem{definition}{Definition}
    \newtheorem*{notation}{Notation}
    \newtheorem{example}{Example}
        \newtheorem*{answer}{Ответ}
    \newtheorem*{draft}{Черновик ответа}

\begin{document}
\textbf{The following is true ONLY for odd $\omega$.\\}
We can generalize the definition of $\phi$ functional:
\begin{definition}
    $$
    \phi(x) = \begin{cases}
                    \omega(x_0), & 0 \le x_0 \le p - 2 \\
                    \phi(\sigma x), & x_0 = p - 1
                \end{cases}
    $$
\end{definition}
    The classic case is hence described by $\omega(j)=j$. This is the only non-trivial $\omega$ for $p=3$. \\
    
\begin{definition}
Let $\phi_\star = \min\limits_x \phi(x), \phi^\star = \max\limits_x \phi(x), \delta=\phi^\star - \phi_\star$. We say that a functional $\phi$  \textit{has the palindromic property} iff 
$$
\forall j,\ 0 \le j \le m\delta:   \pi_m(m\phi_\star + j)=\pi_m(m\phi^\star-j)
$$
\end{definition}
    
    Let us find the sufficient conditions for $\phi$ to have the palindromic property.
    
\begin{definition}
We call a function $\omega$  \textit{binary-antipalindromic} iff
\[\begin{cases}
	\mathrm{Ran }\ \omega = \{0,1\} \\
	\forall j,\ 0 \le j \le p-2: \  \omega(j) = 1 - \omega(p-2-j)
\end{cases}\]
\end{definition}


\begin{example}
The usual $\omega(j)=j$ is binary-antipalindromic for $p = 3$. Indeed,
$$
\omega(0) = 0 = 1 - \omega(1) = 1 - \omega(3 - 2 - 0)
$$
$$
\omega(1) = 1 = 1 - \omega(0) = 1 - \omega(3 - 2 - 1)
$$
\end{example}

\begin{theorem}
If $\omega$ is binary-antipalindromic, the corresponding $\phi$ has the palindromic property.
\end{theorem}

We will need the following lemmas to prove the theorem. (Note that the lemmas generalize Lemma 3.1 and Lemma 3.2 from the article "Weak limits of powers of Chacon's automorphism"\ )

\begin{lemma}
If $\omega$ is binary-antipalindromic, the probability distributions of the random sequences $\{\phi(S^j x)\}_{j \ge 0}$ and $\{ 1 - \phi(S^{-j} x)\}_{j \ge 0}$ are the same.
\end{lemma}
\begin{proof}
Let $x \in \Gamma \setminus \{(p-1,p-1,\ldots)\}$. We
say that $\mathrm{order}(x) = k \ge 0$ if $x_0=x_1=\ldots=x_{k-1}=p-1$ and $x_k \ne p-1$.
Since the first digit in the sequence $(\ldots, x-1,x,x+1)$ follows a periodic pattern
$(\ldots, 0, 1, 2, \ldots,p-2, p-1, 0, 1, 2, \ldots,p-2, p-1, \ldots)$, the contribution of points of order $0$ in the sequence
$\{\phi(S^j x)\}_{j \ge 0}$ provides a sequence of blocks $\omega(0),\omega(1),\ldots, \omega(p-2)$ separated by one symbol
given by a point of higher order. To fill in the missing symbols corresponding to positions $j$ 
such that $\mathrm{order}(x+j) \ge 1$, we observe that, if $x$ starts with a $p-1$, then for all $j \ge 0$, $\phi(x+pj)=\phi(\sigma x + j)$. Hence the missing symbols are given by the symbols $\{\phi(S^j \sigma x)\}_{j \ge 0}$.\\
Let us observe an example for $p=7$ produced by the Legendre symbol-like $\omega(x)=((\frac{x+1}{7}) + 1)/2$ (which is binary-antipalindromic due to the Quadratic reciprocity):\bigskip\\
$\begin{array}{cccccccccccccccccccccccccccl}
1 & 1 & 0 & 1 & 0 & 0 & . & 1 & 1 & 0 & 1 & 0 & 0 &. & 1 & 1 & 0 & 1 & 0 & 0 & . & 1 & 1 & 0 & 1 & 0 & 0 &\leftarrow \text{contribution of order }0 \\
  &   &   &   &   &   & 1 &   &   &   &   &   &   &1 &   &   &   &   &   &   & 0 &   &   &   &   &   &   &\leftarrow \text{contribution of order }1 \\
1 & 1 & 0 & 1 & 0 & 0 & 1 & 1 & 1 & 0 & 1 & 0 & 0 & 1 & 1 & 1 & 0 & 1 & 0 & 0 & 0 & 1 & 1 & 0 & 1 & 0 & 0 &\leftarrow \text{the whole sequence} \\
\end{array}$
\bigskip\\
It is proved by induction on order that, using that $\sigma$ preserves the measure $\lambda$, the probability distributions of the random sequences $\{\phi(S^j x)\}_{j \ge 0}$ and $\{ 1 - \phi(S^{-j} x)\}_{j \ge 0}$ for binary-antipalindromic $\omega$ are the same.
\end{proof}
\begin{lemma}
If $\{\phi(S^j x)\}_{j \ge 0} \overset{d}{=} \{ 1 - \phi(S^{-j} x)\}_{j \ge 0}$, then for any $0 \le j \le m:\ \pi_m(j)=\pi_m(m-j)$
\end{lemma}
\begin{proof}
The coefficient $\pi_m(m-j)$ is equal to the probability to see $(m-j)$ digits equal to 1 in the $m$ consecutive terms in $\{\phi(S^j x)\}_{j \ge 0}$. Thus, $\pi_m(m-j)$ is also equal to the probability to see $j$ digits equal to $0$ in the $m$ consecutive terms. Using Lemma 1, we conclude the statement of this lemma.\\
\end{proof}

To prove the theroem, it is now enough to notice that in Definition 1 for $\phi$ produced by binary-antipalindromic $\omega$ we may substitute $\phi_\star = 0, \phi^\star = 1, \zeta=1$ and hence the palindromic properly is equivalent to $\forall j, 0 \le j \le m:\ \pi_m(j)=\pi_m(m-j)$.
This summarizes the consideration of $\omega$ functions with $\mathrm{Ran }\ \omega = \{0,1\} $. Let us skip to the general construction.

\begin{definition}
We call a function $\omega$  \textit{antipalindromic} iff
\[\begin{cases}
	\mathrm{Ran }\ \omega = \{0,1, \ldots, \zeta\} \\
	\forall j,\ 0 \le j \le p-2: \  \omega(j) = \zeta - \omega(p-2-j)
\end{cases}\]
\end{definition}


\begin{notation}
We further use $[M, N]$ instead of $\{M,M+1, \ldots, N\}$ for shortness.
\end{notation}
Note that given $\omega$ such that $\mathrm{Ran}\ \omega = [0,\zeta]$, we have $\mathrm{Ran}\ \phi = [0,\zeta]$ and $\mathrm{Ran}\ \phi^{(m)} = [0,m\zeta]$

\begin{lemma}
If $\omega$ is binary-antipalindromic, the probability distributions of the random sequences $\{\phi(S^j x)\}_{j \ge 0}$ and $\{ \zeta - \phi(S^{-j} x)\}_{j \ge 0}$ are the same.
\end{lemma}
This lemma is shown similarly to the Lemma 1.\bigskip \\
The following theorem generalizes Theorem 1.
\begin{theorem}
If $\omega$ is antipalindromic, the corresponding $\phi$ has the palindromic property.
\end{theorem}
\begin{proof} In Definition 1, we may substitute $\phi_\star = 0, \phi^\star = \delta = \zeta$. Hence it is enough to show that \[\forall j,\ 0 \le j \le m\zeta: \pi_m(j) = \pi_m(m \zeta - j).\]
$\pi_m(m \zeta - j) = \lambda(\phi^{(m)}(x)=m\zeta-j)=
\\=\sum\limits_{(\phi_1, \ldots, \phi_m)} \lambda\big(\phi(x)=\phi_1, \phi^{(2)}(x) = \phi_2, \ldots, \phi^{(m)}(x)=\phi_m\big) \mathbb{I} (\phi_1 + \ldots + \phi_k = m\zeta - j) = 
\\ = \sum\limits_{(\phi_1, \ldots, \phi_m)} \lambda\big(\phi(x)=\phi_1, \phi^{(2)}(x) = \phi_2, \ldots, \phi^{(m)}(x)=\phi_m\big) \mathbb{I} \big((\zeta-\phi_1) + \ldots + (\zeta-\phi_m) = j\big)$\\
Using Lemma 3, this equals to:\\
$\sum\limits_{(\phi_1, \ldots, \phi_m)} \lambda\big(\phi(x)=\zeta-\phi_m, \phi^{(2)}(x) = \zeta-\phi_{m-1}, \ldots, \phi^{(m)}(x)=\zeta-\phi_1\big) \mathbb{I} \big(\sum\limits_k (\zeta-\phi_k) = j\big) = \left< \psi_k := \zeta - \phi_k \right> = 
\\ =\sum\limits_{(\psi_1, \ldots, \psi_m)} \lambda(\phi(x)=\psi_1, \phi^{(2)}(x) = \psi_2, \ldots, \phi^{(m)}(x)=\psi_m) \mathbb{I} (\psi_1 + \ldots + \psi_m = j) =  \lambda(\phi^{(m)}(x)=j)= \\ =\pi_m(j)$
\end{proof}
We may describe a larger set of functionals $\phi$ having the palindromic property with the use of following lemma.
\begin{lemma}[On the affine transformations]
Let $\omega:[0,p-2] \rightarrow [0, \zeta]$ be antipalindromic, then for any $a > 0, b \ge 0$:\\
$\phi'(x) = \begin{cases}
                    a \omega(x_0) + b, & 0 \le x_0 \le p - 2 \\
                    \phi'(\sigma x), & x_0 = p - 1
                \end{cases}$ is antipalindromic.
\end{lemma}
\begin{proof}
Let $\pi_m'(j) = \lambda(\phi'^{(m)}(x)=j)$. In terms of Definition 1, $\phi_\star = b, \phi^\star = a\zeta + b, \delta = a\zeta$. Let us prove that $\pi_m'(mb + j) = \pi_m' (m(a\zeta+b)-j)$.\\
First, we perform the division with remainder: $j = qa + r$. It follows from the construction of $\phi'$ that $\pi_m'(mb + qa + r) = 0$ if $r \ne 0$. Yet, $m(a\zeta+b)-j = m(a\zeta+b)-qa - r = mb + (m\zeta -q)a - r$ and hence $\pi_m' (m(a\zeta+b)-j) = 0$ if $r \ne 0$.\\
Thus, it remains to prove that $\pi_m'(mb + qa) = \pi_m' (m(a\zeta+b)-qa)$.
Note that we may restore the values of $\phi$ produced by $\omega$ from the values of $\phi'$. Indeed, consider the bijection $i: \{b, a + b, 2a + b \ldots, a\zeta + b\} \rightarrow [0, \zeta]$ such that $i(j) = \frac{j-b}{a}$. It's easy to see that $i(\phi'(x))=\phi(x)$. Subsequently, we may define $i^{(m)}(j) = \frac{j-mb}{m}$ and conclude $i^{(m)}(\phi'^{(m)}(x))=\phi(x)$.\\
Now let us prove $\pi_m'(mb + qa) = \pi_m' (m(a\zeta+b)-qa)$ using the fact that $i^{(m)}$ is bijective.\\
\[\pi_m'(mb + qa) = \lambda(\phi'^{(m)}(x)=mb + qa) = \lambda\big(i^{(m)}(\phi'^{(m)}(x))=i^{(m)}(mb + qa)\big)  =\]\[= \lambda(\phi^{(m)}(x)= \frac{mb-qa-mb}{m})= \lambda(\phi^{(m)}(x)= q)=\pi_m(q)\]
Similarly, $\pi_m' (m(a\zeta+b)-qa) = \pi_m(m\zeta - q)$. Since $\omega$ is antipalindromic, it follows from Theorem 2 that $\pi_m(q) = \pi_m(m \zeta - q)$ and then $\pi_m'(mb + qa) = \pi_m' (m(a\zeta+b)-qa)$.
\end{proof}
\begin{proposition}[On inheritance of polindromic property]
Let $\phi$ have the polindromic property, and let there be $\phi'$ such that there exists bijection $i^{(m)}: \mathrm{Ran}\ \phi'^{(m)} \rightarrow \mathrm {Ran}\ \phi^{(m)}$. Then $\phi'$ has the polindromic property.
\end{proposition}
The proof of this must be the same as for Lemma 4.
\end{document}