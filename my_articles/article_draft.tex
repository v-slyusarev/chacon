

According to the palindromic property (\Cref{pal_property}), the coefficients of $t^j$ and $t^{-j}$ in $\SymPol{p}{m}(t)$ are equal. Hence, we may write it in a compact form: $\SymPol{m}{p}(t) = \sum\limits_{j=0}^{D_m^p} \frac{a_m^p(j)}{2} \left(t^{-\frac{j}{2}} + t^\frac{j}{2}\right)$. When the number of terms is odd, the coefficients with odd indexes are zeroes. When the number of terms is even, the coefficients with even indexes are zeroes.

We may rewrite the symmetric form as polynomial of hyperbolic functions. Indeed, let $t = e^z$, then $\SymPol{m}{p}(e^z) = \sum\limits_{j=0}^{D_m^p} \frac{a_m^p(j)}{2} \left(e^{-\frac{jz}{2}} + e^\frac{jz}{2}\right) = \sum\limits_{j=0}^{D_m^p} a_m^p(j) \ch \frac{jz}{2}$.

Then the recurrence equations take the form:

\begin{align}
    \SymPol{p}{pm}(e^z)&=\SymPol{p}{m}(e^z)\\
    \SymPol{p}{pm+k}(e^z)&=\frac{1}{p}\left((1-(-1)^{p-k-j})\sum\limits_{j=0}^{p-k-1} \ch\frac{jk}{2}\right)\SymPol{p}{m}(e^z)+\frac{1}{p}\left((1-(-1)^{k-j})\sum\limits_{j=0}^{k-1} \ch\frac{j(p-k)}{2}\right)\SymPol{p}{m+1}(e^z)
\end{align}