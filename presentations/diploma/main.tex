% $Header: /cvsroot/latex-beamer/latex-beamer/solutions/conference-talks/conference-ornate-20min.en.tex,v 1.7 2007/01/28 20:48:23 tantau Exp $

\documentclass{beamer}
%\usefonttheme{proffessionalfonts}


% This file is a solution template for:

% - Talk at a conference/colloquium.
% - Talk length is about 20min.
% - Style is ornate.

%\usepackage{mathptmx}
%\usepackage{helvet}

%\usepackage{beamerthemesplit}                                 

%\usepackage{ucs}
\usepackage[cp1251]{inputenc}                                 
\usepackage[english,russian]{babel}
\usepackage[OT1,T2A]{fontenc}

\usepackage{amsmath}
\usepackage{amssymb}
\usepackage{mathrsfs}

\usepackage{graphicx}
\usepackage{grffile}

\usepackage{a}

\def\ASet#1{{\mathbb{#1}}}
\def\conv{\mathbin{*}}

\def\Berns{{\boldsymbol \beta}}

\def\UF{{\mathrm{UF}}}

\def\SIGN#1{{\color{red} #1\/\,}}

\def\Lamp{{\mathscr{L}}}

\definecolor{QuoteColor}{rgb}{0.0,0.0,0.5} 
\definecolor{PointColor}{rgb}{0.7,0.0,0.0} 



\mode<presentation>
{
	%\usecolortheme{beetle}
	%\usecolortheme{crane}
	\usecolortheme{dolphin} % !!!
	%\usecolortheme{seahorse}
	%\usecolortheme{fly}

  \usetheme{Szeged}
  %\usetheme{Warsaw}
  %\usetheme{boxes} %%%
  % or ...

  \setbeamercovered{transparent}
  % or whatever (possibly just delete it)
}


\usepackage[english]{babel}
% or whatever

%\usepackage[utf8x]{inputenc}
% or whatever

%\usepackage{times}
%\usepackage[T1]{fontenc}
% Or whatever. Note that the encoding and the font should match. If T1
% does not look nice, try deleting the line with the fontenc.

\usepackage{cmbright}

%\input ParisTitle.tex
%\input CaenTitle.tex
%\input ETDS14Title.tex
\input title.tex


% If you have a file called "university-logo-filename.xxx", where xxx
% is a graphic format that can be processed by latex or pdflatex,
% resp., then you can add a logo as follows:

% \pgfdeclareimage[height=0.5cm]{university-logo}{university-logo-filename}
% \logo{\pgfuseimage{university-logo}}



% Delete this, if you do not want the table of contents to pop up at
% the beginning of each subsection:
\AtBeginSubsection[]
{
  %\begin{frame}<beamer>{Outline}
  %  \tableofcontents[currentsection,currentsubsection]
  %\end{frame}
}


% If you wish to uncover everything in a step-wise fashion, uncomment
% the following command: 

%\beamerdefaultoverlayspecification{<+->}


\begin{document}

\begin{frame}
  \titlepage
\end{frame}

\begin{frame}{Outline}
  \tableofcontents
%  % You might wish to add the option [pausesections]
\end{frame}


% Structuring a talk is a difficult task and the following structure
% may not be suitable. Here are some rules that apply for this
% solution: 

% - Exactly two or three sections (other than the summary).
% - At *most* three subsections per section.
% - Talk about 30s to 2min per frame. So there should be between about
%   15 and 30 frames, all told.

% - A conference audience is likely to know very little of what you
%   are going to talk about. So *simplify*!
% - In a 20min talk, getting the main ideas across is hard
%   enough. Leave out details, even if it means being less precise than
%   you think necessary.
% - If you omit details that are vital to the proof/implementation,
%   just say so once. Everybody will be happy with that.



% Iceberg 

\def\CAT{{\mathrm{CAT}}}
\def\ATC{{\mathrm{ATC}}}
\def\TCA{{\mathrm{TCA}}}
\def\seppoint{\,{\boldsymbol .}\,}
\def\CATalphabet{\{{\mathrm C}, {\mathrm A}, {\mathrm T}\}}
\def\syC{{\mathrm{C}}}
\def\syA{{\mathrm{A}}}
\def\syT{{\mathrm{T}}}

\def\gCAT{\includegraphics[width=18mm,height=18mm]{cat_0.eps}}
\def\gATC{\includegraphics[width=18mm,height=18mm]{cat_120.eps}}
\def\gTCA{\includegraphics[width=18mm,height=18mm]{cat_240.eps}}

\def\iceberg{{\mathfrak I}}
\def\ipart{\bar\iceberg}
\def\ice{\ipart}
\def\ib{*}
\def\icebody{\iceberg^\ib}

\def\tow{{\mathfrak T}}
\def\twp{{\mathfrak T}}
\def\twpe{{\mathfrak T}^\varepsilon}
\def\utw{\cup\,\tow}

\def\VisualSIGN#1{{\color{red} #1}}
\def\SIGN#1{{\em\color{red} #1\/}}

\def\spMult{{\boldsymbol m}}
\def\Mult{M}

\def\CyR{R^\circ}

\def\term#1{{\large\rm\itshape #1}}
\def\Comment#1{{\footnotesize #1}}




%=================================================================================================
% CONTENT 

\input GenChaconPoly.tex
%\input example.tex

%-------------------------------------------------------------------------------------------------
\begin{frame}

	\begin{center}
	{\huge\color{blue} Thank you for attention!}
	%{\huge\color{blue} ������� �� ��������!}
	\end{center}

\end{frame}	
	
%\fi
\end{document}









